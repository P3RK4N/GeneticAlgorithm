\subsection{\Large Pytorch and Python}
    \textbf{Pytorch}\cite{Stevens} is an open-source machine learning library based on the Torch library. It was developed by Facebook's artificial intelligence research group and was released in 2016. PyTorch is commonly used for applications such as natural language processing and computer vision. It offers a dynamic computational graph, a feature that allows for more flexibility and ease of use when building and training neural networks. Additionally, PyTorch includes support for CUDA, a parallel computing platform and API for GPU-accelerated computing, making it well-suited for training large, deep neural networks on powerful hardware. It has become a popular library in the research community, largely due to its dynamic computation graph and built-in support for CUDA. While primarily developed for GPU, Pytorch enables running the computations on CPU also. In this work we use it to enable fast prototyping and CPU benchmarking.

    \textbf{Python}. PyTorch is a library built for Python. One of the key features of PyTorch is its ability to enable fast iteration times and easy debugging. With PyTorch, developers can easily perform complex operations on tensors (multidimensional arrays) without the need for explicit computation graphs. Additionally, PyTorch's built-in debugging tools, such as the ability to track gradients and perform dynamic graph visualization, make it easier to identify and correct errors in the model. This makes PyTorch a popular choice for researchers and practitioners.

     Both \textbf{Numpy} and PyTorch are powerful libraries for handling multidimensional arrays and performing mathematical operations on them. However, in our application, we chose to use PyTorch because of its native support for neural network operations. In particular, PyTorch provides a convenient module called nn, which includes a wide range of commonly used neural network layers, such as nn.ReLU, that make it easy to build and train complex models. This is the reason we choose Pytorch over Numpy