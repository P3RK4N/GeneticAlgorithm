    In conclusion, this project has successfully implemented a system for visualizing and simulating the movement of shapes using shaders and buffers. Our study has shown that increasing the number of instances in a neural network can result in a faster learning process for the task of simulating walking. However, it also leads to slower computing speeds and longer iteration times. Thus, it is important to strike a balance between the number of instances and the speed of computing in order to optimize the performance of the neural network. Further research can be conducted to identify the optimal number of instances for different types of tasks and computing environments.

\subsection{\LARGE Future work}

    There is still room for improvement in the future. One area for improvement is the physics simulation, which could be optimized to run more efficiently. Another potential improvement is the use of more advanced algorithms for matrix multiplication, such as the Winograd algorithm, which has a lower computational complexity compared to naive matrix multiplication. Additionally, the size of the neural networks used in the genetic computation could be reduced as they are not required to be complex in this project, this will speed up the process and make it more efficient.
